%%%%%%%%%%%%%%%%%%%%%%%%%%%%%%%%%%%%%%%%%%%%%%%%%%%%%%%%%%%%%%%%%%% 
%                                                                 %
%                            CHAPTER                              %
%                                                                 %
%%%%%%%%%%%%%%%%%%%%%%%%%%%%%%%%%%%%%%%%%%%%%%%%%%%%%%%%%%%%%%%%%%% 

\chapter{Literature Review}
    In this chapter we'll go over the state of the art in the field of object counting. We'll go over the techniques commonly used for counting and go in more depth about the topic of few shot object detection and why it should be applied to our problem. Finally we'll go over the metrics used to evaluate the performance of the models.

\section{Counting}
    Counting networks are quite an established concept in machine learning as numerous papers tackle the issue of counting humans, cars, animals or cells. What those have in common is that they only encompass a small set of possible categories to count and that, as they have a large real life use, large annotated datasets exist for these problems like ShanghaiTech and COWC. 

%https://openaccess.thecvf.com/content/CVPR2022W/L3D-IVU/papers/Ranjan_Vicinal_Counting_Networks_CVPRW_2022_paper.pdf

    
    
\section{Proposed approach}
    In this section we'll go into more detail about the approach we'll be taking.
    This will include a number of subsections, each describing a different aspect of the approach.
    \subsection{Few-shot}


