%%%%%%%%%%%%%%%%%%%%%%%%%%%%%%%%%%%%%%%%%%%%%%%%%%%%%%%%%%%%%%%%%%% 
%                                                                 %
%                            CHAPTER                              %
%                                                                 %
%%%%%%%%%%%%%%%%%%%%%%%%%%%%%%%%%%%%%%%%%%%%%%%%%%%%%%%%%%%%%%%%%%% 

\chapter{Litature Review}

\section{State of the art/related work}
    There are plenty of papers about counting the number of objects in an image,
    with plenty of different approaches/architectures. 
    As \url{https://openaccess.thecvf.com/content/CVPR2022W/L3D-IVU/papers/Ranjan_Vicinal_Counting_Networks_CVPRW_2022_paper.pdf}
    situates itself in quite a similar position as to our problem, 
    using its references as a reference could be a good idea.
    
\section{Proposed approach}
    In this section we'll go into more detail about the approach we'll be taking.
    This will include a number of subsections, each describing a different aspect of the approach.
    \subsection{Few-shot}


