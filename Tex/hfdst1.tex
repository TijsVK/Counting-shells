%%%%%%%%%%%%%%%%%%%%%%%%%%%%%%%%%%%%%%%%%%%%%%%%%%%%%%%%%%%%%%%%%%% 
%                                                                 %
%                            CHAPTER                              %
%                                                                 %
%%%%%%%%%%%%%%%%%%%%%%%%%%%%%%%%%%%%%%%%%%%%%%%%%%%%%%%%%%%%%%%%%%% 

\chapter{Introduction}

Every year, one day per year, nearly a thousand volunteers travel to the Belgian coast to collect and categorize the shells that wash up on the beach. This data is collected by the flemish institute for the sea (VLIZ) to study populations of marine molluscs and the impact of their environment(climate change, fishing, etc) on the population. The volunteers participating in this study are mostly enthoustiasts, but also scientists and families with children. To ensure a good quality of the data, most volunteers participate in a workshop. The counting of the shells is done by walking along the beach and noting every shell that is found individually. This is a very time consuming process, and the volunteers are often not very experienced in counting, resulting in mistakes with all but the most common shells. When a volunteer finds a shell that they are not familiar with, they can contact a helpdesk to help them.

**Insert flowchart of usual process of counting here**

The fact that the project relies on volunteers to do most of the legwork, combined with the fact that experts have to man the checkpoints and the helpdesk makes the project unscalable beyond having a single dedicated day per year. With over 5 million people visiting the Belgian coast every year, there is a lot of potential to collect more data if the process of collecting the data could be simplified to be accessible to anyone visiting the beach at any time.

In this thesis, we will attempt to simplify the process of collecting data so that it can be done by anyone, anywhere, at any time.