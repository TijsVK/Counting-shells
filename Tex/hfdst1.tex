%%%%%%%%%%%%%%%%%%%%%%%%%%%%%%%%%%%%%%%%%%%%%%%%%%%%%%%%%%%%%%%%%%% 
%                                                                 %
%                            CHAPTER                              %
%                                                                 %
%%%%%%%%%%%%%%%%%%%%%%%%%%%%%%%%%%%%%%%%%%%%%%%%%%%%%%%%%%%%%%%%%%% 

\chapter{Introduction}

Every year, one day per year, nearly a thousand volunteers travel to the Belgian coast to collect and categorize the shells that wash up on the beach. This data is collected by the flemish institute for the sea (VLIZ) to study populations of marine molluscs and the impact of their environment (climate change, fishing, etc) on the population. The volunteers participating in this study are mostly enthusiasts, but also scientists and families with children. To ensure a good quality of the data, most volunteers participate in a workshop. The counting of the shells is done by walking along the beach and noting every shell that is found individually. This is a very time consuming process, and the volunteers are often not very experienced in counting, resulting in mistakes with all but the most common shells. When a volunteer finds a shell that they are not familiar with, they can contact a helpdesk to help them.

**Insert flowchart of usual process of counting here**

The fact that the project relies on volunteers to do most of the legwork, combined with the fact that experts have to man the checkpoints and the helpdesk, makes the project unscalable beyond having a single dedicated day per year. With over 5 million people
%https://www.kustportaal.be/nl/toerisme-en-recreatie#:~:text=De%20Belgische%20kust%20is%20de,in%20totaal%2027.723.420%20overnachtingen.
visiting the Belgian coast every year, there is a lot of potential to collect more data if the process of collecting the data could be simplified to be accessible to anyone visiting the beach at any time.

In this thesis, we will attempt to simplify the process of collecting data so that it can be done by anyone, anywhere, at any time. We will do this by training a counting network to recognize shells in an image and count them automatically. This is already done on a smaller scale by VLIZ with obsidentify. Obsidentify is a mobile app and website where users can submit pictures of a single shell and get a result of what kind of shell it is. This is a useful tool, but taking a close up picture of every single shell is a very time consuming process. We will have to work with a limited dataset to train the neural network as no dataset exist with large quantities of anotated pictures of beaches. After succesfull completion of this thesis, the new ideal scenario for collecting data can be found in figure \ref{fig:ideal_scenario}. Compared to the current process, found in figure \ref{fig:current_scenario}, this new process nearly elliminates the experts involvement and thus makes the process scalable.
**Insert flowchart of new simplified process of counting here**

We will be studying if current counting networks are performant enough to recognize shells in beach image. We will build up to this by first training a network to count objects from a more established dataset in order to have a baseline to compare our model to. Afterwards we will then train that network with a small dataset to count shells and study its performance. 

In the remainder of this thesis we will first discuss the state of the art in the field of object detection and counting, with a focus on few-shot learning. We will then discuss the datasets and the network architecture we will be using. In the second semester we will implement the network and train it on the datasets. We will then discuss the results and the limitations of our model. Finally we will discuss the future work that can be done to improve the model.