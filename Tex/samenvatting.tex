Every year the Flemisch Institute for the sea organizes a shell counting day to map the diversity of our seaside. 
Thousands of volunteers go to the beach to count and classify shells. 
In this thesis, we will try to automate this process by using neural networks.
The goal is to be able to count the shells in an uncontrolled environment so the volunteers would only have to take pictures of the shells and the neural network would do the rest.

This is not a trivial task, as the shells are not always in the same position, the lighting conditions are not always the same and the shells are not always the same size.
The large variety of shells, with some of them being very similar, makes it even harder to detect them correctly.

We are also limited in the amount of data we can use to train our neural network as we do not have a large annotated dataset of shells. We will thus have to use a few-shot approach to train our network.